\documentclass{article}

\usepackage[colorlinks=true, allcolors=blue]{hyperref}
\usepackage{booktabs}

\title{\vspace{-3em} VND \\ Luís Felipe Ramos Ferreira \vspace{-3em}}
\date{}

\begin{document}

\maketitle

Como heurística VND foi desenvolvido o algoritmo de \href{https://en.wikipedia.org/wiki/Simulated_annealing}{\textit{Simulated Annealing}}. Essa metaheurísticas é inspirada
no processo físico de \textit{annealing} utilizado na metalurgia pra alterar as propriedades físicas de algum metal. Essa abordagem pode ser utilizada em diversos problemas de natureza difícil,
inclusive no TSP. O código da implementação pode ser encontrado nesse \href{https://github.com/lframosferreira/tsp-heuristics}{repositório}, em particular a heurística descrita foi implementada no arquivo
\textit{simulated\_annealing.py}.

\begin{center}
	\begin{table}[h!]
		\centering
		\begin{tabular}{|c|c|c|c|}
			\hline
			Algorithm & Instance     & Path Weight & Time (s) \\ \hline
			sa        & kroD100.tsp  & 32445       & 0.1660   \\ \hline
			sa        & lin105.tsp   & 21930       & 0.1835   \\ \hline
			sa        & att48.tsp    & 11916       & 0.0681   \\ \hline
			sa        & kroB150.tsp  & 47631       & 0.2201   \\ \hline
			sa        & berlin52.tsp & 9419        & 0.0708   \\ \hline
			sa        & kroA200.tsp  & 58072       & 0.2929   \\ \hline
			sa        & rat195.tsp   & 4146        & 0.2827   \\ \hline
			sa        & pr152.tsp    & 148754      & 0.2201   \\ \hline
			sa        & st70.tsp     & 880         & 0.0967   \\ \hline
			sa        & pr144.tsp    & 132181      & 0.2077   \\ \hline
			sa        & pr124.tsp    & 123801      & 0.1748   \\ \hline
			sa        & pr136.tsp    & 159539      & 0.1975   \\ \hline
			sa        & kroB200.tsp  & 62789       & 0.2937   \\ \hline
			sa        & kroB100.tsp  & 33466       & 0.1424   \\ \hline
			sa        & kroA150.tsp  & 48895       & 0.2198   \\ \hline
			sa        & kroA100.tsp  & 35865       & 0.1425   \\ \hline
			sa        & pr107.tsp    & 75366       & 0.1497   \\ \hline
			sa        & kroE100.tsp  & 34696       & 0.1406   \\ \hline
			sa        & rat99.tsp    & 1786        & 0.1378   \\ \hline
			sa        & kroC100.tsp  & 31168       & 0.1420   \\ \hline
			sa        & pr76.tsp     & 142017      & 0.1082   \\ \hline
		\end{tabular}
		\caption{Path weights and times for various TSP instances using the sa (simulated annealing) algorithm}
	\end{table}

\end{center}


\end{document}
