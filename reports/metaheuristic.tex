\documentclass{article}

\usepackage[colorlinks=true, allcolors=blue]{hyperref}
\usepackage{booktabs}

\title{\vspace{-3em} Metaheurística \\ Luís Felipe Ramos Ferreira \vspace{-3em}}
\date{}

\begin{document}

\maketitle

Como metaheurística foi desenvolvido o algoritmo de \href{https://en.wikipedia.org/wiki/Simulated_annealing}{\textit{Simulated Annealing}}. Essa metaheurísticas é inspirada
no processo físico de \textit{annealing} utilizado na metalurgia pra alterar as propriedades físicas de algum metal. Essa abordagem pode ser utilizada em diversos problemas de natureza difícil,
inclusive no TSP. O código da implementação pode ser encontrado nesse \href{https://github.com/lframosferreira/tsp-heuristics}{repositório}, em particular a heurística descrita foi implementada no arquivo
\textit{simulated\_annealing.py}.

A ideia principal da abordagem é testar caminhos novos aleatoriamente a cada iteração. Se o caminho novo for melhor que o atual, ele é mantido. Se for pior, ele é mantido com uma probabilidade proporcional à hiperparâmetros
do algoritmo. O objetivo é fazer com que o algoritmo busque no espaço de soluções a solução ótima e sempre se movimente em direção à um mínimo. O uso de processos estocásticos auxilia para evitar que o algoritmo
fique preso em um mínimo local e possa explorar novas áreas, podendo chegar assim mais perto do mínimo global.

Como hiperparâmetros inicias para essa bateria de testes apresentada neste documento, foram utilizados os valores de temperatura inicial igual a 10, \(\delta_t = 0.9\) e limite inferior de temperatura igual a 0.10. Esse valores foram interesantes
pois apresentaram um bom resultado em um tempo não muito alto para as instâncias testadas. No mundo real, diferentes valores devem ser testados para garantir que o objetivo principal ao ser utilizado o algoritmo seja atingido.
Se o interesse principal for uma execução rápida, os parâmetros inciais podem ser bem diferentes de casos em que o desejo é o de um caminho do caixeiro de custo baixo, em troca de uma execução lenta.

\begin{center}
	\begin{table}[h!]
		\centering
		\begin{tabular}{|c|c|c|c|}
			\hline
			Algorithm & Instance     & Path Weight & Time (s) \\ \hline
			sa        & kroD100.tsp  & 32445       & 0.1660   \\ \hline
			sa        & lin105.tsp   & 21930       & 0.1835   \\ \hline
			sa        & att48.tsp    & 11916       & 0.0681   \\ \hline
			sa        & kroB150.tsp  & 47631       & 0.2201   \\ \hline
			sa        & berlin52.tsp & 9419        & 0.0708   \\ \hline
			sa        & kroA200.tsp  & 58072       & 0.2929   \\ \hline
			sa        & rat195.tsp   & 4146        & 0.2827   \\ \hline
			sa        & pr152.tsp    & 148754      & 0.2201   \\ \hline
			sa        & st70.tsp     & 880         & 0.0967   \\ \hline
			sa        & pr144.tsp    & 132181      & 0.2077   \\ \hline
			sa        & pr124.tsp    & 123801      & 0.1748   \\ \hline
			sa        & pr136.tsp    & 159539      & 0.1975   \\ \hline
			sa        & kroB200.tsp  & 62789       & 0.2937   \\ \hline
			sa        & kroB100.tsp  & 33466       & 0.1424   \\ \hline
			sa        & kroA150.tsp  & 48895       & 0.2198   \\ \hline
			sa        & kroA100.tsp  & 35865       & 0.1425   \\ \hline
			sa        & pr107.tsp    & 75366       & 0.1497   \\ \hline
			sa        & kroE100.tsp  & 34696       & 0.1406   \\ \hline
			sa        & rat99.tsp    & 1786        & 0.1378   \\ \hline
			sa        & kroC100.tsp  & 31168       & 0.1420   \\ \hline
			sa        & pr76.tsp     & 142017      & 0.1082   \\ \hline
		\end{tabular}
		\caption{Resultados médios para as execuções}
	\end{table}

\end{center}


\end{document}
