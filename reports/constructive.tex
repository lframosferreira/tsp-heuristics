\documentclass{article}

\usepackage[colorlinks=true, allcolors=blue]{hyperref}
\usepackage{booktabs}

\title{Heurística Construtiva}
\author{Luís Felipe Ramos Ferreira}
\date{\href{mailto:lframos.lf@gmail.com}{\texttt{lframos.lf@gmail.com}}
}

\begin{document}

\maketitle

Como heurística construtiva foram desenvolvidos tanto o algoritmo \textit{twice around the tree} quanto o algoritmo de \(Christofides\).
No entanto, para simplificar, abordarei nesse relatório apenas o primeiro. O código fonte pode ser encontrado neste \href{https://github.com/lframosferreira/tsp-heuristics}{repositório}, em particular
o código do \textit{twice artound the tree} pode ser encontrado no arquivo \texttt{algorithms.py}.

O algoritmo \textit{twice around the tree} é um algoritmo construtivo e aproximativo para o problema do caixeiro viajante simétrico e que respeita a desigualdade triangular. O algoritmo é 2-aproximativo, ou seja,
encontra um caminho do caixeiro que tem comprimento no máximo duas vezes igual ao caminho ótimo.

A tabela abaixo mostra o tempo médio requerido por cada instância quando executadas 20 vezes cada uma. A tabela também mostra o custo do caminho do caixeiro encontrado pelo algoritmo.

\begin{center}
	\begin{tabular}{lllrr}
		\toprule
		instance     & path\_weight & time     \\
		\midrule
		kroD100.tsp  & 27112        & 0.005099 \\
		lin105.tsp   & 19495        & 0.006008 \\
		kroB150.tsp  & 36150        & 0.012731 \\
		berlin52.tsp & 10114        & 0.001455 \\
		kroA200.tsp  & 40028        & 0.023495 \\
		rat195.tsp   & 3234         & 0.022293 \\
		pr152.tsp    & 87995        & 0.012656 \\
		st70.tsp     & 888          & 0.002403 \\
		pr144.tsp    & 80599        & 0.011018 \\
		pr124.tsp    & 74139        & 0.008042 \\
		pr136.tsp    & 151904       & 0.010339 \\
		kroB200.tsp  & 40703        & 0.022531 \\
		kroB100.tsp  & 25885        & 0.006176 \\
		kroA150.tsp  & 35119        & 0.012949 \\
		kroA100.tsp  & 27210        & 0.005451 \\
		pr107.tsp    & 54237        & 0.006193 \\
		kroE100.tsp  & 29965        & 0.005831 \\
		rat99.tsp    & 1693         & 0.004874 \\
		kroC100.tsp  & 27968        & 0.005468 \\
		pr76.tsp     & 145336       & 0.002865 \\
		\bottomrule
	\end{tabular}
\end{center}



\end{document}
